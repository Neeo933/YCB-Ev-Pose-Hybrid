%!TEX encoding = UTF-8 Unicode

%---------------------------------------------------------------%
%  使用XeLaTex编译
%  参考文献的排版,请创建 .bib 文件, 并使用 BibTex 或者 Biber(有中文参考文献时) 进行排版。
%---------------------------------------------------------------%

%===============================================================%
%  固定模板,请不要修改这一部分

\documentclass[a4paper,11pt,onecolumn,UTF8]{article}
\usepackage{CVClassTemplate}  
\usepackage{booktabs}  % 用于 \toprule, \midrule, \bottomrule
\usepackage{array}     % 表格格式支持                                
\setmainfont{Times New Roman}                      

\newcommand{\mysecondauthor}{null}
\newcommand{\mythirdauthor}{null} 
\newcommand{\myfourthauthor}{null}
\newcommand{\myfifthauthor}{null}
%===============================================================%


%===============================================================%
% ** 请从这里开始你的编辑 **

%---------------------------------------------------------------%
%  基本信息配置
%---------------------------------------------------------------%
\newcommand{\mytitle}          % 请输入论文题目
{面向复杂环境与高速运动场景的事件驱动6D物体姿态估计}

\newcommand{\myfirstauthor}        % 请输入作者姓名
{\kaishu 南晓璐}

\newcommand{\myfirstaffiliation}   % 请输入作者单位
{\small 清华大学航天航空学院}

\newcommand{\myfirstemail}         % 请输入作者电子邮箱
{\small nxl25@mails.tsinghua.edu.cn}

\renewcommand{\mysecondauthor}{\kaishu 许嘉杰}    % 若有两名作者取消该行注释
\newcommand{\mysecondaffiliation}{\small 清华大学航天航空学院}
\newcommand{\mysecondemail}{\small xujiajie25@mails.tsinghua.edu.cn}

%\renewcommand{\mythirdauthor}{\kaishu 第三作者}    % 若有三名作者取消该行注释
\newcommand{\mythirdaffiliation}{\small 单位3}
\newcommand{\mythirdemail}{\small xxxx@xxxx.xxx}

%---------------------------------------------------------------%
%  摘要 & 关键词
%---------------------------------------------------------------%
\newcommand{\myabstract}       % 请在下面输入中文摘要
{
    在太空非合作目标抓取、高速工业流水线等极端视觉场景中,传统视觉方法面临巨大挑战。RGB相机在强光直射下易过曝,在高速相对运动下产生运动模糊,导致6D姿态估计失效作为神经形态传感器,事件相机异步记录亮度变化,并将这些变化以稀疏事件流的形式呈现,具有高时间分辨率和高动态范围的优势。为了解决这一问题,本文提出了一种融合事件相机数据的几何引导多模态网络 —— \textbf{GMG-PVNet}。

    该方法包含三个核心创新:(1) 引入 \textbf{MTS (Motion-Texture Surface)} 表征,将高频事件流转化为纹理特征,捕捉不受光照影响的运动边缘;(2) 设计 \textbf{AFDM (Attentive Feature Diffusion Module)},利用稀疏的事件点云引导密集图像特征的增强,解决了稀疏-密集模态融合的难题;(3) 提出 \textbf{Gated Fusion} 机制,自适应地融合纹理与几何特征。实验结果表明,该方法在YCB-Video数据集的单物体模型上实现了超过90\%的ADD-S精度,在多物体复杂场景下,针对对称物体仍保持了鲁棒的几何对齐能力,证明了事件驱动视觉在极端条件下的有效性。
}

\newcommand{\mykeywords}        % 请在下面输入中文关键词,以中文分号分隔
{
    6D姿态估计;事件相机;多模态融合;AFDM;MTS
}

%---------------------------------------------------------------%
%  预置宏包和自定义命令(你可以补充需要的宏包和自定义命令)
%---------------------------------------------------------------%
\usepackage[pdfborder=0,CJKbookmarks=true]{hyperref}  % 使用内部超链接,其中第二个选项用于支持中文书签
\hypersetup{
    hidelinks=true,
}
%改变颜色

\addbibresource{example.bib}				% 填入参考文献库文件.bib

%===============================================================%
%  打印标题页信息,请不要修改这一部分
\begin{document}
\printtitlepage
%===============================================================%

%---------------------------------------------------------------%
%  正文内容从这里开始
%---------------------------------------------------------------%
\section{引言}
\subsection{研究背景与挑战}
随着机器人技术向太空探索和高速自动化领域延伸,视觉系统的工作环境变得日益严苛。例如,在太空碎片清理任务中,目标物体通常处于极端的明暗变化中(一面被太阳直射过曝,一面全黑);在高速机械臂抓取任务中,rgb相机快速运动会导致成像模糊。

传统的6D姿态估计方法(如PoseCNN~\parencite{xiang2017posecnn}, PVNet~\parencite{peng2019pvnet})主要依赖RGB图像的纹理特征,在上述场景下极易失效。虽然RGB-D方法(如DenseFusion~\parencite{wang2019densefusion})引入了深度信息,但在物体边缘处,深度传感器常产生跳变噪声,且无法提供高频的运动信息。

\textbf{事件相机 (Event Camera)} 作为一种生物启发式传感器,具有微秒级的时间分辨率和极高的动态范围(HDR)。它仅记录亮度的变化,天然对运动边缘敏感,且不受运动模糊影响。这为解决极端场景下的姿态估计问题提供了新的契机。
\subsection{本文贡献}
为了充分利用事件相机的优势,本文提出了GMG-PVNet,主要贡献如下:
\begin{enumerate}
    \item \textbf{数据表示创新:MTS (Motion-Texture Surface)}。我们将时间维度的异步事件流编码为空间纹理,生成MTS图像。这不仅保留了高频运动信息,还解决了传统CNN无法直接处理异步数据的难题。
    \item \textbf{架构创新:AFDM模块}。针对3D事件点云(稀疏)与2D图像特征(密集)难以对齐的问题,我们提出了“注意力特征扩散模块”。它利用PointNet提取的全局几何特征作为“锚点”,通过注意力机制激活2D特征图的边缘响应。
    \item \textbf{融合机制:自适应门控融合}。设计了Gated Fusion模块,使网络能够根据输入质量(如RGB是否模糊),自适应地加权纹理特征与几何特征。
\end{enumerate}

\section{相关工作}

\subsection{6D物体姿态估计}
6D物体姿态估计旨在从图像中恢复物体的三维位置和旋转。基于RGB的方法通常通过回归2D关键点并利用PnP算法求解姿态,如PVNet~\parencite{peng2019pvnet}提出的像素级投票网络,显著提升了遮挡下的鲁棒性。基于RGB-D的方法如DenseFusion~\parencite{wang2019densefusion},通过像素级融合RGB和深度特征,在纹理弱的物体上表现更好。然而,这些方法在高速运动或极端光照下,受限于传感器的物理特性,往往难以提取有效的特征。

\subsection{基于事件视觉的应用}
事件相机在光流估计、SLAM和边缘检测等领域已展现出巨大潜力。然而,将其应用于6D物体姿态估计的工作相对较少。现有的方法多是将事件流转换为体素网格(Voxel Grid)或时间面(Time Surface)作为CNN输入,忽略了事件点云本身包含的精细几何结构。

本文尝试结合MTS的纹理优势与原始点云的几何优势,实现更鲁棒的姿态估计。

\section{方法}

GMG-PVNet采用三流(Three-stream)并行架构,分别处理RGB图像、几何信息(Depth + MTS)以及原始事件点云。整体架构如图~\ref{fig:arch}所示。

\begin{figure}[htbp]
    \centering
    \includegraphics[width=0.9\linewidth]{fig/framework.png} % 请替换为你的架构图   
    \caption{GMG-PVNet 整体架构图。包含RGB分支、几何分支(MTS+Depth)和PointNet分支,通过Gated Fusion和AFDM模块进行多模态融合。}
	\label{fig:arch}
\end{figure} 

\subsection{多模态数据流}

\subsubsection{RGB Stream}
使用ResNet-18作为骨干网络,提取物体的颜色和语义纹理特征。这一分支在光照条件良好且物体静止时起主导作用。

\subsubsection{MTS Stream (Motion-Texture Surface)}
在处理事件相机的数据时,最常用的一种办法是将时空邻域内的事件通过简单方式(例如,通过像素级的事件计数或极性累加)转换为二维图像。事件帧的优势非常明显,主要因为:(i)它们是将陌生的事件流转换为熟悉二维表示的简单方法,该表示包含场景边缘的空间信息(这是自然图像中最具信息量的区域),(ii)它们不仅传递事件存在的信号,也传递事件缺失的信息(同样具有信息价值),(iii)其具有直观的解释(例如边缘图、亮度增量图像),以及(iv)它们是兼容传统计算机视觉的数据结构~\parencite{gallego2020event}。
但上述方法忽略了事件数据的高时间分辨率特性,导致图像被压缩,丢失了时间分辨率的信息。

因此,为了利用事件相机的高频特性,我们引入MTS表征。利用指数衰减函数,将一段时间 $\Delta t$ 内的事件流 $(x, y, t, p)$ 投影到2D平面。像素亮度 $I(x,y)$ 定义为:
\begin{equation}
    I(x, y) = \sum_{e_i \in \mathcal{E}(x,y)} p_i \cdot e^{-(t_{now} - t_i) / \tau}
\end{equation}
MTS相当于一个“时域高通滤波器”,自动过滤了静止背景,仅保留物体的运动轮廓。在光照剧烈变化或RGB模糊时,MTS仍能提供清晰的边缘纹理。同时替代传统的单色直方图。不仅解决了运动模糊问题,并为网络提供了显式的运动导数。MTS处理效果如图~\ref{fig:MTScomparison}所示。

\begin{figure}[htbp]
    \centering
    \includegraphics[width=0.9\linewidth]{fig/MTScomparison.png} % 请替换为你的架构图   
    \caption{MTS 处理效果对比图。(a)原始事件帧(b)MTS 图像(c)深度图(d)MTS+Depth 拼接图。}
	\label{fig:MTScomparison}
\end{figure} 


\subsubsection{Geometry Stream}
我们将深度图(Depth)与MTS在通道维度拼接,通过共享权重的ResNet提取物体的空间几何特征,如~\ref{fig:guided_6d_convincing_results}所示。

\begin{figure}[htbp]
    \centering
    \includegraphics[width=0.9\linewidth]{fig/guided_6d_convincing_results.png}    
    \caption{融合图。(a)MTS输入图像(b)深度图梯度(c)自适应空间域筛选(d)MTS边缘 (e)融合几何特征(f)深度与MTS融合表示。}
	\label{fig:guided_6d_convincing_results}
\end{figure} 

\subsection{AFDM:稀疏点云引导密集特征}
这是本文的核心创新。为了利用事件数据的微秒级精度,我们不仅使用了MTS图像,在这里还直接加入了原始事件点云。AFDM(Attentive Feature Diffusion Module)旨在解决稀疏点云与密集特征图的融合问题。

\begin{enumerate}
    \item \textbf{特征提取}:使用轻量级PointNet处理在线采样(Online Sampling)得到的 $N=1024$ 个事件点,提取包含精细时空信息的局部几何特征 $F_{point} \in \mathbb{R}^{N \times C}$。
    \item \textbf{空间散射 (Spatial Scattering)}:利用事件点的 $(x, y)$ 像素坐标,将PointNet提取的高维特征精准地“散射”回2D特征图的对应位置,形成稀疏特征图 $M_{sparse}$。
    \item \textbf{注意力扩散 (Attentive Diffusion)}:使用卷积层作为“扩散器”,将这些稀疏的几何特征向周围像素传播,从而“点亮”特征图中的物体边缘。
\end{enumerate}

\subsection{推理与几何修正}

\subsubsection{RANSAC Voting}
网络输出像素级向量场,表示每个像素指向物体关键点的单位向量。我们采用RANSAC算法聚类出9个2D关键点(8个角点+1个中心点),有效抵抗遮挡和离群点干扰。

\subsubsection{深度修正 (Depth-Guided Refinement)}
由于单目PnP对Z轴深度的估计存在尺度漂移,我们引入了几何约束策略。利用网络预测的Mask区域,统计深度图的中位数作为真实的Z轴距离。这是一种有效的传感器融合(Sensor Fusion)手段,显著提升了定位精度。

\section{实验}

\subsection{实验设置}
我们在YCB-Ev SD数据集上进行了评估。该数据集包含21类物体,是一个用于6DoF目标姿态估计的标准清晰度(SD)分辨率事件相机数据的合成数据集~\parencite{rojtberg2025ycb}。评价指标采用 \textbf{ADD-S}(针对对称/几何形状)和 \textbf{ADD}(针对非对称/纹理)。

\subsection{核心能力验证:单物体高精度实验}
为了验证模型在理想情况下的性能上限,我们选取了具有代表性的物体(如 \texttt{003\_cracker\_box})进行独立训练与测试。

结果显示,在单物体实验中,GMG-PVNet实现了超过 \textbf{90\%} 的ADD-S准确率($<10\%$直径误差)。这证明了在特征专注的情况下,引入事件和几何信息能极高精度地恢复物体的3D结构。MTS提供的强边缘特征使得关键点预测不再“缩成一团”,而是精准分布在物体角点。


\begin{figure}
    \centering
    \includegraphics[width=0.8\linewidth]{fig/singleobj.png} % 请替换为你的Loss曲线图
    \caption{单物体实验结果。GMG-PVNet在单物体场景下实现了超过90\%的ADD-S准确率,证明了事件数据在高精度姿态估计中的有效性。}
    \label{fig:single_obj_results}
\end{figure}

\subsection{复杂场景分析:多物体实验}
我们进一步测试了模型在21类物体混合场景下的泛化能力。表~\ref{tab:multi_obj}展示了部分物体的评估结果。

\begin{table}[htbp]
    \centering
    \caption{多物体场景下的评估结果(部分展示)}
    \label{tab:multi_obj}
    \begin{tabular}{lccc}
        \toprule
        \textbf{Object Type} & \textbf{Metric} & \textbf{Accuracy ($<10\%d$)} & \textbf{Mean Error} \\
        \midrule
        \textbf{Symmetric Objects} (e.g., Bowl, Clamp) & \textbf{ADD-S} & \textbf{40\% - 60\%} & \textbf{$< 20$ mm} \\
        Asymmetric Objects (e.g., Mustard) & ADD & $\sim 10\%$ & $\sim 100$ mm \\
        \bottomrule
    \end{tabular}
\end{table}

\textbf{结果解读:}
\begin{itemize}
    \item \textbf{几何定位强:} 在对称物体上,ADD-S分数较高,说明模型成功捕捉到了物体的几何形状和空间位置。
    \item \textbf{纹理歧义:} 在非对称物体上,ADD分数较低。可视化分析发现,这是由于ResNet-18容量限制及纹理特征被弱化,导致模型出现了180°的旋转歧义(即头尾颠倒),但包围盒的重合度依然很高。
\end{itemize}

\subsection{消融实验}
为了验证多模态融合的必要性,我们对比了“纯RGB”与“多模态融合(Ours)”的效果。

\begin{figure}[htbp]
    \centering
    % \includegraphics[width=0.8\linewidth]{fig/ablation.pdf} % 请替换为你的Loss曲线图
    \caption{消融实验对比。引入事件点云后,模型的平均几何误差显著降低。}
	\label{fig:ablation}
\end{figure} 

对比结果表明,引入MTS和事件点云后,模型的平均几何误差(Mean ADD Error)显著降低。在定性分析中,当RGB图像因光照较暗导致纹理不清时,纯RGB模型往往无法检测到物体;而GMG-PVNet凭借MTS提供的清晰轮廓,依然能准确框出物体位置。

\subsection{定性结果展示}
图~\ref{fig:vis}展示了模型在复杂场景下的推理结果。绿色框为模型预测的3D包围盒。可以看出,即使在存在遮挡和背景杂波的情况下,模型依然能够准确地定位物体。

\begin{figure}[htbp]
    \centering
    % \includegraphics[width=1.0\linewidth]{fig/vis_result.jpg} % 请替换为你的可视化结果图
    \caption{GMG-PVNet 在YCB-Video数据集上的定性结果。绿色框为预测结果,蓝色框为真值。}
	\label{fig:vis}
\end{figure} 

\section{总结与未来展望}

本文针对极端光照和高速运动场景,提出了一种基于事件相机的6D姿态估计网络GMG-PVNet。通过MTS表征和AFDM模块,我们成功地将高频事件信息融合进姿态估计流程。

\textbf{主要结论:} (1) 事件数据有效性:MTS和事件点云显著增强了模型对物体边缘的感知能力,特别是在几何定位精度上表现优异。(2) 鲁棒性:通过深度修正和动态掩码策略,模型在复杂背景下展现了极强的抗干扰能力。

\textbf{局限与展望:} 目前模型在多物体分类和纹理方向识别上仍有提升空间。未来工作将集中在引入更大容量的Backbone以解决纹理混淆问题,并加入基于RGB的Refinement网络以消除旋转歧义。


%---------------------------------------------------------------%
%  参考文献
%---------------------------------------------------------------%
\printbibliography

%===============================================================%

\end{document}
